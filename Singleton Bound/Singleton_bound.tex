\documentclass[12pt]{article}
\usepackage{amsmath, amssymb, amsthm}
\usepackage{geometry}
\geometry{a4paper, margin=1in}

\begin{document}

\title{Understanding the Singleton Bound in Coding Theory}
\author{}
\date{}
\maketitle

\section*{Theorem 2.1: Singleton Bound}
Let \( C \) be a linear code of length \( n \), dimension \( k \), and minimum distance \( d \) over \( \mathbb{F}_q \). Then:
\[
d \leq n - k + 1
\]

This theorem establishes an upper bound for the minimum distance of a linear code given its length and dimension.

\subsection*{Key Insights}
- The minimum distance \( d \) is the smallest number of positions in which any two codewords differ.
- For Reed-Solomon codes \( RS(k, q) \), the minimum distance is exactly \( n - k + 1 \). Codes that meet the Singleton Bound are called \textit{Maximum Distance Separable (MDS)} codes.

\section*{Proof of the Singleton Bound}
We define a subset \( W \subseteq \mathbb{F}_q^n \) as:
\[
W := \{ \mathbf{a} = (a_1, \dots, a_n) \in \mathbb{F}_q^n \mid a_d = a_{d+1} = \dots = a_n = 0 \}.
\]
For any \( \mathbf{a} \in W \), the weight \( \text{wt}(\mathbf{a}) \leq d - 1 \). Therefore, \( W \cap C = \{0\} \).

We know:
\[
\dim(W + C) = \dim W + \dim C,
\]
where \( W + C := \{\mathbf{w} + \mathbf{c} \mid \mathbf{w} \in W, \mathbf{c} \in C \} \).

Since \( \dim W = d - 1 \) and \( \dim C = k \), this implies:
\[
d - 1 + k \leq n.
\]
Thus:
\[
d \leq n - k + 1.
\]

\section*{Implications of the Singleton Bound}
1. The Singleton Bound is true for both linear and non-linear codes. 
2. Codes that achieve the bound are called \textbf{MDS codes}.

\subsection*{Practical Limitations}
Reed-Solomon codes are a restrictive class of MDS codes due to their alphabet size requirements. They are not suitable for all cases, especially when \( q = 2 \).

\subsection*{Generalization}
For any code of length \( n \), with \( M \) codewords and minimum distance \( d \) over an alphabet of size \( q \), we have:
\[
M \leq q^{n - d + 1}.
\]

\section*{Conclusion}
The Singleton Bound plays a foundational role in coding theory, guiding the design and analysis of error-correcting codes. While the bound is tight for MDS codes, finding codes that approach this bound with practical efficiency remains an open challenge.

\end{document}

